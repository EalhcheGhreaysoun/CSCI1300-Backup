\documentclass{article}
\usepackage[utf8]{inputenc}
\usepackage[a4paper, total={6in, 8in}]{geometry}


\begin{document}
CSCI 1300 Recitation HW5

\vspace{1cm}

Question 1:
\begin{verbatim}
        #include <iostream> 
        using namespace std;

        int main() {

        int num = 1; 

        while(num <= 10) {
            cout << num << " ";
            num = num + 1;     // removed the int before num
        } 

        return 0; 
        } 
\end{verbatim}

\vspace{1cm}
Question 2:
\begin{verbatim}
        #include <iostream> 
        using namespace std;

        int main() { 

        int even = 2;

        while (even <= 9) { //changed != to <=
            cout << even << " "; 
            even += 2; 
        } 

        return 0; 
        } 
\end{verbatim}

\vspace{1cm}

Question 3:

the psudocode outputs 3, -3, 6, -6, 9.

\vspace{2cm}

Question 4:

\hspace{0.25cm}4.a:

\begin{verbatim}
        define function applyDeductions
        pass in income
        pass in number of dependants
        if income is less than 14600:
        return 0;
        else:
        if number of dependants is greater than zero
        return income - 14600 - 500
        else
        return income -14600
        end

        define function calculateTax
        pass in income
        if income <= 10000:
        return income * 0.9
        else if income <= 50000: 
        return income * 0.85
        else: 
        return income * 0.8
        end

        define function computeNetIncome
        pass in gross income, number of dependants
        declare double variable money
        set money to calculateTac(applyDeductions(income, number of dependants))
        if number of dependants is greater than zero: 
        return money + 14600 + 500
        else: 
        return money + 14600
        end
\end{verbatim}

\vspace{0.25cm}
\hspace{0.25cm}4.b:

    example input 1: income: \$100,000, dependants: 2, output: \$83,020

    example input 2: income: \$10,000, dependants: 0, output: \$10,000

    \hspace{0.25cm}4.c:

\begin{verbatim}
    assert(computeNetIncome(100000, 2) == 83020)
    assert(computeNetIncome(10000, 0) == 10000)
\end{verbatim}

\vspace{0.25cm}

\hspace{0.25cm}4.d:

\begin{verbatim}
        #include <iostream>
        #include <cassert>
        
        using namespace std;
        
        double applyDeductions(double income, int numDependents) {
            if(income < 14600){
                return 0;
            }else{
                if(numDependents >=0){
                    return income - 14600 -500;
                }else{
                    return income -14600;
                }
            }
        }
        
        double calculateTax(double taxableIncome) {
            if(taxableIncome <= 10000){
                return taxableIncome * 0.9;
            } else if (taxableIncome <= 50000){
                return taxableIncome * 0.85;
            } else{
                return taxableIncome * 0.8;
            }
        }
        
        double computeNetIncome(double grossIncome, int numDependents) {
            double money = calculateTax(applyDeductions(grossIncome, numDependents));
            if (money == 0){
                return grossIncome;
            }else{
                if(numDependents > 0){
                    return money + 14600 +500;
                }else{
                    return money + 14600;
                }
            }
        }
        
        int main(){
            assert(computeNetIncome(100000, 2) == 83020);
            assert(computeNetIncome(10000, 0) == 10000);
        }
\end{verbatim}

\end{document}