\documentclass{article}
\usepackage[utf8]{inputenc}
\usepackage[a4paper, total={7in, 9in}]{geometry}
\usepackage{listings}


\begin{document}
CSCI 1300 Recitation HW 15

\vspace{1cm}

Question 1:

\begin{lstlisting}[language=C++]
    #include <vector>
    #include <iostream>
    using namespace std;
    
    int main() {
        vector<int> v = {5, 6, 7, 8, 9};
        //added find and v.at to iterate to the correct element
        v.erase(find(v.begin(), v.end(), v.at(0)));
        
        cout << v[0] << endl;
        return 0;
    }
    
\end{lstlisting}

\vspace{1cm}

Question 2:

\begin{lstlisting}[language=C++]
    #include <vector>
    #include <iostream>
    using namespace std;
    
    int main() {
        vector<int> v;
        //used vector push_back instead of vector at
        v.push_back(42);
        cout << v.at(0) << endl;
        return 0;
    }
    
\end{lstlisting}

\vspace{1cm}

Question 3:

\begin{lstlisting}[language=C++]
    #include <vector>
    #include <iostream>
    using namespace std;
    
    int main() {
        vector<int> v = {2, 4, 6, 8, 9};
    
        for (int i = 0; i < v.size(); i++) {
            if ((v.at(i) % 2) == 0) {
                v.erase(v.begin() + i);
                i--; //decrement to deal with decreasing size of v
            }
        }
    
        // Print remaining elements
        for (int i = 0; i < v.size(); i++) {
            cout<< v.at(i) <<" ";
        }
        cout << endl;
        return 0;
    }    
\end{lstlisting}

\vspace{1cm}

Question 4:

\begin{lstlisting}[language=C++]
    #include <vector>
    #include <iostream>
    using namespace std;
    
    int main() {
        vector<int> v = {5, 6, 7, 8, 9};
        
        int length = v.size(); //static length variable
    
        for(int i = 0; i < length; i++){
            v.push_back(v[i] * 2);
        }
        
        for(int i=0; i<v.size(); i++){
            cout<< v[i]<< endl;
        }
    }
    
\end{lstlisting}

\vspace{1cm}

Question 5:

\begin{lstlisting}[language=C++]
    /*
    Psudocode:
    define function Fibonacci
    take input n
    variable total = 1
    variable prev_total = 0
    for and integer i = 0 less than n:
    x = total;
    total = prev_total+x;
    prev_total = x;
    increment i
    end
    end
    
    inputs:
    n=3, total = 2
    n=5, total = 5
    n=8, total = 21
    */
    #include <iostream>
    
    using namespace std;
    
    int Fibonacci(int n){
        int total = 1;
        int prev_total = 0;
        int x = 0;
    
        for (int i = 0; i < n-1; i++){
            x = total;
            total = prev_total+x;
            prev_total = x;
        }
        return total;
    }
    int main(){
        cout << Fibonacci(8) << endl;
        return 0;
    }
\end{lstlisting}

\end{document}