\documentclass{article}
\usepackage[utf8]{inputenc}
\usepackage[a4paper, total={6in, 8in}]{geometry}
\usepackage{listings}



\begin{document}
CSCI 1300 Recitation HW5

\vspace{1cm}

Question 1:

\begin{verbatim}
    #include <iostream>
    using namespace std;

    int main() {

        int x = 3, y = 4;

        for (int i = y; i >= 0; i--) {
            for (int j = x; j >= 0; j--) {
                cout << "(" << j << ", " << i << ")  "; //switched i and j to give (x, y)
            }
            cout << endl;
        }
    }
\end{verbatim}

\vspace{1cm}

Question 2:

\begin{verbatim}
    #include <iostream>
    using namespace std;
    
    int main() {
    
        int numbers[5] = {10, 20, 30, 40, 50};
    
        cout << "The contents of the array are: ";
    
        for (int i = 0; i <= 5; i++) { //removed = in <=
            cout << numbers[i] << " "; //added index to numbers
        }
    
        cout << endl;
        int sum = 0; //moved the initialization of sum outside the for loop
        for (int i = 0; i <= 5; i++) { //removed = in <=
            sum += numbers[i]; //need to have index for arrays
        }
    
        cout << "Sum = " << sum << endl;
        return 0;
    }
    
\end{verbatim}

\vspace{1cm}

Question 3:


\begin{lstlisting}[language=C++]
    #include <iostream>
    #include <cmath>
    using namespace std;
    
    bool isPrime(int num) {
    
        if (num <= 1) { //added curly braces for readability
            return false;
        }
        for (int i = 2; i <= sqrt(num); i++) { //< becomes <= to account for perfect squares
            if (num % i == 0) { //added ==
                return false;
            }
        }
    
        return true;
    }
    
    int main(){
    
        int num = 37;
        if (isPrime(num)) {
            cout << num << " is a prime number." << endl;
        } else {
            cout << num << " is not a prime number." << endl;
        }
    
        return 0;
    }
\end{lstlisting}

\vspace{1cm}

Question 4:

\begin{lstlisting}[language=C++]
    #include <iostream>
    #include <string>
    using namespace std;
    
    int main()
    {
        string languages[5] = {"C++", "Python", "Java", "Matlab", "Julia"}; //changed the array length to 5
        int product = 1; //changed initial product to 1
        int total = 5;
    
        for (int i = 0; i < total; i++){ //removed = in <=
            product *= languages[i].length(); //added parentheses to length 
        }
    
        cout << "Product of lengths = " << product << endl;
        return 0;
    }
    
\end{lstlisting}

\vspace{1cm}

Question 5:
\hspace{0.25cm}5.a:
\begin{verbatim}
    define integer array set1 of size 3
    define integer array set2 of size 2
    get user input and set set1 to the input
    get user input and set set3 to the input
    define an integer i and set it to 0;
    while i is less than the size of set1:
    define an integer j and set it to 0;
    while j is less than the size of set2:
    output "Dish " set[i] + " with Drink " + set[2]
    end
    end
\end{verbatim}

\hspace{0.25cm}5.b:
\begin{lstlisting}{language=C++}
    #include <iostream>
    #include <string>
    using namespace std;
    
    int main()
    {
        int set1[3];
        int set2[2];
        string input;
        cout << "Please enter 3 codes for the main dishes:" << endl;
        cin >> set1[0];
        cin >> set1[1];
        cin >> set1[2];
    
        cout << "Please enter 2 codes for the drinks:" << endl;
        cin >> set2[0];
        cin >> set2[1];
    
        cout << "Here are the available meal combinations:" << endl;
        for (int i = 0; i < 3; i++){
            for(int j = 0; j < 2; j++){
                cout << "Dish " << set1[i] << " with Drink " << set2[j] << endl;
            }
        }
    
    }
\end{lstlisting}

the code output for the above code is as follows:

\begin{verbatim}
    Please enter 3 codes for the main dishes:
    1 2 3
    Please enter 2 codes for the drinks:
    10 20
    Here are the available meal combinations:
    Dish 1 with Drink 10
    Dish 1 with Drink 20
    Dish 2 with Drink 10
    Dish 2 with Drink 20
    Dish 3 with Drink 10
    Dish 3 with Drink 20
\end{verbatim}
\end{document}