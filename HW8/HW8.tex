\documentclass{article}
\usepackage[utf8]{inputenc}
\usepackage[a4paper, total={6in, 8in}]{geometry}
\usepackage{listings}


\begin{document}
CSCI 1300 Recitation HW 8

\vspace{1cm}

Question 1:

\begin{lstlisting}[language=C++]
    #include <iostream>
    #include <string> 
    using namespace std;
    int main()
    {
        int size = 7; //changed size to 7
        //cannot init array during run time
        double scores[7] = {85.4, 90.3, 100, 89, 74.5, 95.0, 82.3};
        double sum = 0;
        for(int i = 0; i < size; i++)
        {
            sum += scores[i];
        }
        int avg = sum / size; //changed 6.0 to size
        cout << "Average = " << avg << endl;
        return 0; 
    }
\end{lstlisting}

\vspace{1cm}

Question 2:

\begin{lstlisting}[language=C++]
    #include <iostream>
    using namespace std;
    
    // Function to calculate the transpose of a matrix
    void transposeMatrix(int matrix[][3], int n, int m) 
    { 
        for (int i = 0; i < n; i++)
        {
            for (int j = i + 1; j < m; j++)
            {
                int temp[n][m]; //must define matrix before used
                temp[i][j] = matrix[i][j];
                matrix[i][j] = matrix[j][i];
                matrix[j][i] = temp[i][j];
            }
        }
    } 
    
    int main() 
    {
        const int rows = 3;
        const int cols = 3;
        int originalMatrix[rows][cols] = 
        {
            {1, 1, 1},
            {2, 2, 2},
            {3, 3, 3}
        };
    
        // Calculate the transpose matrix using the function
        transposeMatrix(originalMatrix, rows, cols); //pass by reference
    
        // Display the transpose matrix
        cout << "Transpose Matrix:" << endl;
        for (int i = 0; i < rows; i++) 
        {
            for (int j = 0; j < cols; j++) 
            {
                //cout original matrix
                cout << originalMatrix[i][j] << " "; 
            }
            cout << endl;
        }
    
        return 0;
    }
\end{lstlisting}

\vspace{1cm}

Question 3:

\begin{lstlisting}[language=C++]
    #include <iostream>
    using namespace std;

    int main()
    {
        int N = 4; //N=4
        //changed array type to string
        string item[] = {"book", "pen", "pencil", "eraser"}; 

        //printing all the items
        for (int i = 0; i < N; i++)
        {
            //changed index in item to i
            cout << "The item list has " << item[i] << endl; 
        }
        return 0;
    }
\end{lstlisting}

\vspace{1cm}

Question 4:

\begin{lstlisting}[language=C++]
    #include <iostream>
    #include <string>
    using namespace std;

    int main() 
    {
        const int N = 6;
        string animals[N] = {"lion", "cat", "bear", "dog", "elephant", "fox"};
        for (int i = 0; i < N; i++) 
        {
            if (animals[i].length() == 4) //changed location of [i]
            {
                cout << animals[i] << endl;
            }
        }
        return 0;
    }
\end{lstlisting}

\vspace{1cm}

Question 5:

\hspace{0.5cm}5.a

\begin{verbatim}
    declare function matrixSum that takes two integer matricies a, b
    declare an integer matrix temp of size 2, 3
    set temp to a+b
    set a to temp
    end

    declare main function
    declare matrix m1 of size 2, 3
    declare matrix m2 of size 2, 3
    output "Enter values for matrix 1, one row at a time:"
    take input into matrix m1
\end{verbatim}

\begin{lstlisting}[language=C++]
    #include <iostream>
    #include <string>
    using namespace std;

    void matrixSum(int a[2][3], int b[2][3]){
        for (int i = 0; i < 2; i++){
            for (int j = 0; j < 3; j++){
                int temp[2][3];
                temp[i][j] = a[i][j]+b[i][j];
                a[i][j] = temp[i][j];
            }
        }
    }


    int main() 
    {
        int m1[2][3];
        int m2[2][3];
        cout << "Enter values for matrix 1, one row at a time:" << endl;
        cin >> m1[0][0];
        cin >> m1[0][1];
        cin >> m1[0][2];

        cin >> m1[1][0];
        cin >> m1[1][1];
        cin >> m1[1][2];

        cout << "Enter values for matrix 2, one row at a time: " << endl;
        cin >> m2[0][0];
        cin >> m2[0][1];
        cin >> m2[0][2];

        cin >> m2[1][0];
        cin >> m2[1][1];
        cin >> m2[1][2];

        matrixSum(m1, m2);
        cout << "The sum is: " << endl;
        for (int i = 0; i <2; i++){
            for(int j = 0; j < 3; j++){
                cout << m1[i][j] << " ";
            }
            cout << "\n";
        }
    }
\end{lstlisting}

\vspace{1cm}
coding telephone:

there was not enough time for everyone to do their part of the code. below is the code from James Benish-Kingsbury, the last person to do it in our group.

\begin{verbatim}
    void suggestActivity(string[] ,int num );
    cout << "Day" << num << arr[num] << endl;

    int main()
    int integer = 3;
    string type;
\end{verbatim}

\end{document}